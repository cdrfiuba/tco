\documentclass[a4paper,12pt]{article}   %Hoja A4, tamaño de letra 12 puntos
\usepackage[latin1]{inputenc}        	%Encoding del archivo de texto plano (txt), así podés escribir letras con diacríticas 
\usepackage[T1]{fontenc}        	%Encoding de la fuente
\usepackage[spanish]{babel}         	%Estilos, hyphenation y nombres de entornos (abstract-->resumen) en español
\usepackage{times}            		%Para agregar la fecha al documento         
\usepackage{amsmath}            	%Simbolos matemáticos que sino no funcionan
\usepackage{amssymb}                 	%Simbolos matemáticos que sino no funcionan
\usepackage{amsfonts}                	%Fuentes extras para modo mátemático
\usepackage{units}                   	%Para agregar unidades y fracciones con formato especial
\usepackage{esint}                   	%Para integrales dobles cerradas
%\usepackage{wasysym}                	%Para integrales dobles cerradas (no tan lindo como las de `esint')
%\usepackage{pdfpages}            	%Para incluir otros pdfs, en un nuestro documento Latex
%\usepackage{eso-pic}
%\usepackage{everyshi}
%\usepackage{ifthen}
%\usepackage{calc}
%\usepackage{watermark}              	%Permite incluir marcas de agua
\usepackage{float}                   	%Permite fijar una figura a un lugar específico del archivo con el parámetro [H] 
\usepackage{fancyhdr}            	%Manejo enriquecido de encabezados y pie de página
\usepackage{fancybox}                	%Permite agregar marcos especiales en las páginas
%\usepackage{anysize}            	%Habilita el comando \marginsize{left}{right}{top}{bottom}
%\usepackage{mathptmx}            	%Tipografía Times
\usepackage[hang,bf]{caption2}        	%Numerar subsection
\usepackage{subfigure}
\usepackage{graphicx}
%\usepackage[pdflatex]{graphicx}    	%Permite incluir imágenes en el documento. Usando ``dvipdfm'' se pueden manejar archivos jpg, jpeg, png, y pdf. Pero hay que generar un archivo .bb que contiene el tamaño de la imagen entre otros datos. Para eso miktex-->bin-->ebb.exe 
\usepackage{threeparttable}        	%Permite agregar notas al pie en tablas
\usepackage{cancel}            		%Permite tachar cosas en modo matemático
                    			%hyperref posibilita referencia internas y externas, a continuacion las opciones de config:
\usepackage[a4paper,
             dvipdfm,                 	%Usar sólo si después se va a pasar de dvi a pdf
        %dvips,%        		%Usar si se va a trabajar en dvi, o si después se va a pasar de dvi a ps (y eventualmente de ps a pdf ---pero ojo que los enlaces quedan más lindos si se pasa directamente de dvi a pdf, usando dvipdfm)
        pdfdisplaydoctitle=true,	%Muestra el título en el marco superior de la ventana del visor de pdf
        colorlinks=true,         	%Al encender esta función, ya nohay recuadros feos alrededor de los links
        pdfnewwindow=true,       	%Para que el link a nuevos pdf, no se abran en la misma ventana
        linkcolor=black,         	%Color para links internos
        citecolor=green,         	%Color para links a Bibliografia
               filecolor=blue,          %Color para links a archivos
        pdftitle={Proyecto (66.09) [Junio 2010] - Robótica Colaborativa},
        pdfauthor={Manuel I. Fernandez, Marcela F. Luberto y Sebastián García Marra}]
        {hyperref}
 
\usepackage{color}                 
\usepackage{listings}        		%Este paquete es para poder insertar codigo en el informe y su respectivo set
\lstset{ 
language=C++,                    	% choose the language of the code
basicstyle=\footnotesize,       	% the size of the fonts that are used for the code
numbers=left,                   	% where to put the line-numbers
numberstyle=\footnotesize,      	% the size of the fonts that are used for the line-numbers
stepnumber=1,                   	% the step between two line-numbers. If it is 1 each line will be numbered
numbersep=5pt,                  	% how far the line-numbers are from the code
backgroundcolor=\color{white},  	% choose the background color. You must add \usepackage{color}
showspaces=false,               	% show spaces adding particular underscores
showstringspaces=false,         	% underline spaces within strings
showtabs=false,                 	% show tabs within strings adding particular underscores
frame=single,                 		% adds a frame around the code
tabsize=2,                 		% sets default tabsize to 2 spaces
captionpos=b,               		% sets the caption-position to bottom
breaklines=true,            		% sets automatic line breaking
breakatwhitespace=false,        	% sets if automatic breaks should only happen at whitespace
escapeinside={\%*}{*)}          	% if you want to add a comment within your code
}
                     
 
 
\newcommand{\autor}[3]{\vspace{5mm}%
                                            {\bf #1}\\%
                                            {\bf Padrón:} #2\\%
                                            \texttt{#3}\\}
 
 
%Determinación del estilo de página (encabezado y pie)
 
\lhead{\small Proyecto (66.09) [Junio 2013]}
\chead{}
\rhead{\small TCO}
\rfoot{}
\cfoot{{\thepage}}
\lfoot{}
 
%Manejo de márgenes
 
%\marginsize{left}{right}{top}{bottom}
\addtolength{\textwidth}{2cm}          %Agrega 2cm al ancho por defecto (usado junto con hoffset, no se pierde el centrado)
\addtolength{\hoffset}{-1cm}           %Quita 1cm al margen, pero mantiene la relación repecto de cada margen
\addtolength{\textheight}{2cm}
\addtolength{\voffset}{-1cm}
\setlength{\headheight}{14pt}
 
 
%Manejo de contadores (counters)
 
%\setcounter{tocdepth}{2}        	%Profundidad de la aparición de las secciones en el índice:
                   	 		%-1 Part, 0 Chapter, 1 Section, 2 Subsection, 3 Subsubsection, 4 Titled
 
\setcounter{secnumdepth}{1}          	%Profundidad en la enumeración de las secciones:
					%-1 Part, 0 Chapter, 1 Section, 2 Subsection, 3 Subsubsection, 4 Titled paragraph, 5 Titled subparagraph %paragraph, 5 Titled subparagraph
 
 
%==================================== Comienzo del documento ===================================
 
\begin{document}     
                     
%========================================== Caratula ===========================================
 
\pagestyle{fancy}
 
     
%===============================================================================================
 
%=========================================== Presentación ======================================
\newpage
\section{Codigo fuente}
\subsection{main.c}

\begin{lstlisting}

#include "main.h"

//Delacaracion de variables globales
volatile uint8_t    value, flag_piso_blanco;

//Variables encoders
volatile uint32_t   cuenta_encoder_izquierda, cuenta_encoder_derecha;

//Variables sensores pared
volatile uint8_t    sensor_active, status_flag;
volatile uint32_t   interrupciones_timer_1, distancia;


int main(void)
{
    usart_init();		//Inicializacion de UART para enviar/recibir datos por serie.
    sei();			//Activacion de las interrupciones.


    inicializar_puertos_sensores_pared();
    inicializar_sensores_piso();
    inicializar_encoders();     //Al inicializar los encoders lo que se hace es configurar el TIMER 0 y el TIMER 2
    inicializar_puertos_motores();
    inicializar_PWM();		//Al inicializar el PWM lo que se hace es configurar el TIMER 1

    cuenta_encoder_derecha = 0;
    cuenta_encoder_izquierda = 0;
    _delay_ms(500);


	for (;;){

        //Se corrige el rumbo, midiendo la distancia a la pared derecha.
        motores_corregir_rumbo();

        //Se avanza una distancia muy pequena.
        motores_avanzar(200,200, 40);

        //Se mide a la derecha para saber si hay pared o no.
        distancia = medicion_distancia_pared(SENSOR_PARED_DER);

        //Si se entra aca es que no hay pared a la derecha, entonces, se giro a la derecha
        if(distancia > DISTANCIA_GRANDE){

            //Se revisa si el piso es de color blanco, para saber si llego al final del laberinto
            //Para esto, primero se encienden los sensores del piso y se mide. Si hay suelo blanco,
            //se avanza y se mide de nuevo. Si hay doble confirmacion, entonces estoy en la casilla final.
            encender_sensores_piso();

            _delay_us(100);

            if((PINA & (1 << PA2)) == (1 << PA2)){

                motores_avanzar(200,200,200);

                motores_detener();

                _delay_ms(10000);

            }

            //Si se llega a este punto, el suelo es color negro.
            apagar_sensores_piso();

            motores_avanzar(200,200, 200);

            motores_rotar_der_90_grados();

            motores_avanzar(200,200, 450);

        }

        //Si se entra aca, es que hay una pared a la derecha
        else{

            distancia = medicion_distancia_pared(SENSOR_PARED_CEN);

            //Si se entra acá, es que hay pared a la derecha, y al frente
            if(distancia < DISTANCIA_CHICA){

                motores_rotar_izq_90_grados();

                _delay_ms(10);

                distancia = medicion_distancia_pared(SENSOR_PARED_CEN);

                _delay_ms(10);

                //Si se entra aca, es que hay pared a la derecha, al frente y a la izquierda.
                //Entonces rota 180 grados.
                //La distancia tiene que se mayor a distancia chica, porque a veces no queda centrado en la casilla
                if(distancia < 0x00001800){

                    motores_rotar_izq_90_grados();

                }

                motores_avanzar(200,200, 450);
            }
        }
    }

	return 0;
}


//Inicializacion puerto serie
void usart_init()
{
	UCSRB |= (1<<RXCIE) | (1 << RXEN) | (1 << TXEN);	//Encender recepcion transmicion

	UCSRC |= (1 << URSEL) | (1 << UCSZ0) | (1 << UCSZ1); 	//Configuro para usar 8 caracteres

	UBRRL = BAUD_PRESCALE; 				        // Cargo la parte baja del registro

	UBRRH = (BAUD_PRESCALE >> 8); 			        // Cargo la parte alta del registro
}

//Interrupcion puerto serie
ISR (USART_RXC_vect){

    //Echo: Esta funcion recibe un dato por el puerto serie y lo reenvia. Sirve para probar que la conexion esta OK
	value = UDR;    // Tomo el valor recibido, y lo cargo en la variable value
	UDR = value;	// Cargo el buffer con lo almacenado en la variable value
}


//Interrupcion Timer 1
ISR (TIMER1_OVF_vect){

    if (sensor_active)
        interrupciones_timer_1++;

	if(interrupciones_timer_1 > MAX_INTERRUPCIONES_SENSOR_DISTANCIA)
        status_flag = 1;
}

//Interrupcion INT 0
ISR (INT0_vect)
{
    cuenta_encoder_derecha++;

}

//Interrupcion INT 1
ISR (INT1_vect)
{
    cuenta_encoder_izquierda++;

}


//interrupcion sensores de piso

ISR (INT2_vect){

    if((PINA & (1 << PA2)) == (1 << PA2))

       flag_piso_blanco = TRUE;

    //Finalmente hacemos un clear al registro GIFR de interrupcion como se pide en hoja de datos [1].
    GIFR &= ~(1<<INTF2);

}

\end{lstlisting}
 
\subsection{main.h}

\begin{lstlisting}
#ifndef _MAIN_H_
#define _MAIN_H_

#include <avr/io.h>
#include <avr/interrupt.h>
#include <util/delay.h>
#include "pwm.h"
#include "motores.h"
#include "sensores_pared.h"
#include "sensores_piso.h"
#include "encoders.h"

#define USART_BAUDRATE 	9600
#define BAUD_PRESCALE 	(((F_CPU / (USART_BAUDRATE * 16UL))) - 1)

#define MAX_STRING	25

#define DISTANCIA_CHICA   0x00000E00
#define DISTANCIA_GRANDE  0x00003000//0x8C

#define MAX_TIME    100000

#define MAX_INTERRUPCIONES_SENSOR_DISTANCIA 1000


#endif /* _MAIN_H_ */

//Definicion de prototipos
void usart_init(void);
\end{lstlisting}

\subsection{motores.c}
\begin{lstlisting}
#include "motores.h"

extern volatile uint32_t     cuenta_encoder_derecha, cuenta_encoder_izquierda;


void inicializar_puertos_motores(void){

	//Configuro pines como salida
	DDRD |= MOTOR_DER_DIRECTION;
	DDRB |= MOTOR_DER_BRAKE;

	DDRB |= MOTOR_IZQ_BRAKE;
    	DDRD |= MOTOR_IZQ_DIRECTION;

    	//Detengo los motores una vez que inicializo los puertos.
    	motores_detener();
}



void motores_detener(void){

	//Freno: poner brake en alto
	PORTB |= MOTOR_DER_BRAKE;	//Pongo en alto el pin BRAKE del motor 1
	PORTB |= MOTOR_IZQ_BRAKE;	//Pongo en alto el pin BRAKE del motor 2
}



void motores_retroceder(uint8_t velocidad_izquierda, uint8_t velocidad_derecha){

	//Avance sentido 1: PWM H - Direction H - Brake L

	PORTD |= MOTOR_DER_DIRECTION; 	//Pongo en alto el pin DIRECTION del motor 1
	PORTB &= ~MOTOR_DER_BRAKE;	    //Pongo en bajo el pin BRAKE del motor 1

	PORTD &= ~MOTOR_IZQ_DIRECTION; 	//Pongo en bajo el pin DIRECTION del motor 2
	PORTB &= ~MOTOR_IZQ_BRAKE;	    //Pongo en bajo el pin BRAKE del motor 2

    variar_PWM(velocidad_izquierda, velocidad_derecha);

}


void motores_avanzar(uint8_t velocidad_izquierda, uint8_t velocidad_derecha, uint32_t cantidad_cuentas){

	uint8_t  flag_fin_avance = FALSE;
    uint32_t cuenta_encoder_derecha_anterior;

    cuenta_encoder_derecha = 0;
    cuenta_encoder_izquierda = 0;
    cuenta_encoder_derecha_anterior = 0;

    variar_PWM(velocidad_izquierda, velocidad_derecha);

    PORTD &= ~MOTOR_DER_DIRECTION; 	//Pongo en bajo el pin DIRECTION del motor 1
	PORTB &= ~MOTOR_DER_BRAKE;	    //Pongo en bajo el pin BRAKE del motor 1

	PORTD |= MOTOR_IZQ_DIRECTION; 	//Pongo en alto el pin DIRECTION del motor 2
	PORTB &= ~MOTOR_IZQ_BRAKE;	    //Pongo en bajo el pin BRAKE del motor 2


    while(flag_fin_avance == FALSE){

        if((cuenta_encoder_derecha_anterior + 35) >= cuenta_encoder_derecha){

            cuenta_encoder_derecha_anterior = cuenta_encoder_derecha;

            if(cuenta_encoder_derecha < cuenta_encoder_izquierda){

                if(velocidad_derecha < 240)
                    variar_PWM(velocidad_izquierda, velocidad_derecha += 5);

                }

            if(cuenta_encoder_derecha > cuenta_encoder_izquierda){

                if(velocidad_derecha > 130)
                    variar_PWM(velocidad_izquierda, velocidad_derecha -= 5);

            }
        }

        if(cantidad_cuentas < cuenta_encoder_derecha){

            PORTB |= MOTOR_DER_BRAKE;	//Pongo en alto el pin BRAKE del motor 1
            PORTB |= MOTOR_IZQ_BRAKE;	//Pongo en alto el pin BRAKE del motor 2


            flag_fin_avance = TRUE;

        }
    }
}



void motores_rotar_der_90_grados(void){

    uint8_t  flag_fin_rotacion = FALSE;
    uint32_t cuenta_inicial_encoder_derecho     =   cuenta_encoder_derecha;
    uint32_t cuenta_inicial_encoder_izquierdo   =   cuenta_encoder_izquierda;

    variar_PWM(210, 210);

    PORTD &= ~MOTOR_DER_DIRECTION; 	//Pongo en bajo el pin DIRECTION del motor 1
    PORTB &= ~MOTOR_DER_BRAKE;	    //Pongo en bajo el pin BRAKE del motor 1

    PORTD &= ~MOTOR_IZQ_DIRECTION; 	//Pongo en bajo el pin DIRECTION del motor 2
    PORTB &= ~MOTOR_IZQ_BRAKE;	    //Pongo en bajo el pin BRAKE del motor 2

    while(flag_fin_rotacion == FALSE){

        if(cuenta_encoder_derecha >= (cuenta_inicial_encoder_derecho + 360)){

            if(cuenta_encoder_izquierda >= (cuenta_inicial_encoder_izquierdo + 360)){

                PORTB |= MOTOR_DER_BRAKE;	//Pongo en alto el pin BRAKE del motor 1
                PORTB |= MOTOR_IZQ_BRAKE;	//Pongo en alto el pin BRAKE del motor 2

                flag_fin_rotacion = TRUE;

            }
        }
    }
}


void motores_rotar_izq_90_grados(void){

    uint8_t  flag_fin_rotacion = FALSE;
    uint32_t cuenta_inicial_encoder_derecho     =   cuenta_encoder_derecha;
    uint32_t cuenta_inicial_encoder_izquierdo   =   cuenta_encoder_izquierda;

    variar_PWM(210, 210);

    PORTD |= MOTOR_DER_DIRECTION; 	//Pongo en alto el pin DIRECTION del motor 1
    PORTB &= ~MOTOR_DER_BRAKE;	    //Pongo en bajo el pin BRAKE del motor 1

    PORTD |= MOTOR_IZQ_DIRECTION; 	//Pongo en alto el pin DIRECTION del motor 2
    PORTB &= ~MOTOR_IZQ_BRAKE;	    //Pongo en bajo el pin BRAKE del motor 2

    while(flag_fin_rotacion == FALSE){

        if(cuenta_encoder_derecha >= (cuenta_inicial_encoder_derecho + 370)){

            if(cuenta_encoder_izquierda >= (cuenta_inicial_encoder_izquierdo + 370)){

                //Antes tenia 300

                PORTB |= MOTOR_DER_BRAKE;	//Pongo en alto el pin BRAKE del motor 1
                PORTB |= MOTOR_IZQ_BRAKE;	//Pongo en alto el pin BRAKE del motor 2

                flag_fin_rotacion = TRUE;

            }
        }
    }
}


void motores_corregir_rumbo (void){

    uint32_t distancia;
    uint8_t  flag_fin_rotacion = FALSE;

    cuenta_encoder_derecha = 0;
    cuenta_encoder_izquierda = 0;

    _delay_ms(10);

    distancia = medicion_distancia_pared(SENSOR_PARED_DER);

    _delay_ms(10);

    if(distancia < 0x00000650){

        variar_PWM(200, 200);

        PORTD |= MOTOR_DER_DIRECTION; 	//Pongo en alto el pin DIRECTION del motor 1
        PORTB &= ~MOTOR_DER_BRAKE;	    //Pongo en bajo el pin BRAKE del motor 1

        PORTD |= MOTOR_IZQ_DIRECTION; 	//Pongo en alto el pin DIRECTION del motor 2
        PORTB &= ~MOTOR_IZQ_BRAKE;	    //Pongo en bajo el pin BRAKE del motor 2

        while(flag_fin_rotacion == FALSE){

            if(cuenta_encoder_derecha >= 15){

                if(cuenta_encoder_izquierda >= 15){

                    flag_fin_rotacion = TRUE;

                }
            }
        }
    }

    if(distancia > 0x00000750){

        variar_PWM(200, 200);

        PORTD &= ~MOTOR_DER_DIRECTION; 	//Pongo en bajo el pin DIRECTION del motor 1
        PORTB &= ~MOTOR_DER_BRAKE;	    //Pongo en bajo el pin BRAKE del motor 1

        PORTD &= ~MOTOR_IZQ_DIRECTION; 	//Pongo en bajo el pin DIRECTION del motor 2
        PORTB &= ~MOTOR_IZQ_BRAKE;	    //Pongo en bajo el pin BRAKE del motor 2

        while(flag_fin_rotacion == FALSE){

            if(cuenta_encoder_derecha >= 15){

                if(cuenta_encoder_izquierda >= 15){

                    flag_fin_rotacion = TRUE;

                }
            }
        }
    }
}


\end{lstlisting}

\subsection{motores.h}
\begin{lstlisting}
#ifndef MOTORES_H_
#define MOTORES_H_
#include <util/delay.h>

#include <avr/io.h>
#include <avr/interrupt.h>

#include "main.h"
#include "pwm.h"
#include "sensores_pared.h"

#define MOTOR_DER_BRAKE		(1<<PB3)
#define MOTOR_IZQ_BRAKE		(1<<PB4)
#define MOTOR_DER_DIRECTION	(1<<PD7)
#define MOTOR_IZQ_DIRECTION	(1<<PD6)

#define TRUE    1
#define FALSE   0

#endif /* MOTORES_H_ */


//Definicion de prototipos
void inicializar_puertos_motores	(void);
void motores_avanzar			(uint8_t, uint8_t, uint32_t);
void motores_retroceder			(uint8_t, uint8_t);
void motores_detener			(void);
void motores_rotar_der_90_grados    	(void);
void motores_rotar_izq_90_grados	(void);
void motores_corregir_rumbo         	(void);
extern void inicializar_PWM         	(void);
extern void variar_PWM              	(uint8_t, uint8_t);
\end{lstlisting}

\subsection{encoders.c}
\begin{lstlisting}

#include "encoders.h"

void inicializar_encoders(void){

    //Configuro el encoder usando el pin como entrada
    DDRD    &= ~(1<<PD3);   //Configuro como entrada
    PORTD   &= ~(1<<PD3);   //Desactivo el pull up

    //Configuro el encoder usando el pin como entrada
    DDRD    &= ~(1<<PD2);   //Configuro como entrada
    PORTD   &= ~(1<<PD2);   //Desactivo el pull up

    GICR    |= (1<<INT0);   //Activo INT0
    GICR    |= (1<<INT1);   //Activo INT1
   // GICR &= ~(1<<INT2);   //Desactivo INT2

    MCUCR   |= ((1<<ISC01) | (1<<ISC00));     //Configuro la INT0 por flanco ascendente
    MCUCR   |= ((1<<ISC11) | (1<<ISC10));     //Configuro la INT1 por flanco ascendente

}

\end{lstlisting}

\subsection{encoders.h}
\begin{lstlisting}
#ifndef _ENCODER_H_
#define _ENCODER_H_


#include <avr/io.h>
#include <util/delay.h>
#include <avr/interrupt.h>


#endif /* _ENCODER_H_ */

//Definicion de prototipos
void inicializar_encoders(void);
\end{lstlisting}

\subsection{pwm.c}
\begin{lstlisting}

#include "pwm.h"

void inicializar_PWM(void){

	TCCR1A = ((1 << COM1A1)|(1 << COM1A0)|(1 << COM1B1)|(1 << COM1B0)|(0 << WGM11)|(1 << WGM10));
	//COM1A1=1, COM1A0=1, COM1B1=1, COM1B0=1, WGM11=0, WGM10=1
	//Configuracion del PWM como fast PWM de 8 bits

	TCCR1B = ((0 << WGM13)|(1 << WGM12));
	//Timer Apagado WGM13=0, WGM12=1, CS12=0, CS11=0, CS10=0

	// Configuracion de direccion de puertos I/O
	// DDRA=0x00; // Puerto A como entrada
	//DDRC=0xFF; // Puerto C como salida
	//DDRD=0xFF;// Pin 3 y 4 del puerto B como salida
	DDRD |= (1<<4)|(1<<5);
	//PORTA=0xFF; // Habilitacion de pull-ups del puerto A
	//PORTC=0x00; // Inicializacion de salida del puerto C
	//OCR1AH = 0;//velocidad_motor_izq;
	//OCR1BH = 0;//velocidad_motor_der;

    TIMSK |= (1<<TOIE1);
	// Configuracion del Timer 1:
	TCCR1B |= ((0<<CS12)|(0<<CS11)|(1<<CS10)); //Enciendo el Timer sin prescaler (001)
}

void variar_PWM(uint8_t valor_izq, uint8_t valor_der){

	OCR1AL = valor_izq;
	OCR1BL = valor_der;
	//TCNT1 = 0;
}
\end{lstlisting}

\subsection{pwm.h}
\begin{lstlisting}

#ifndef _PWM_H_
#define _PWM_H_

#include <avr/io.h>
#include <util/delay.h>

#endif


//Definicion de prototipos
void inicializar_PWM    (void);
void variar_PWM         (uint8_t valor_izq, uint8_t valor_der);
\end{lstlisting}

\subsection{sensoresPared.c}
\begin{lstlisting}

#include "sensores_pared.h"

extern volatile uint8_t     status_flag, sensor_active;
extern volatile uint32_t    interrupciones_timer_1;

void inicializar_puertos_sensores_pared(void){

	//Configuro los puertos de TRIGGER como salida.
	DDRA |= SENSOR_PARED_DER;
	DDRA |= SENSOR_PARED_CEN;
	DDRA |= SENSOR_PARED_IZQ;

	//Configuro el puerto de ECHO como entrada.
	DDRB &= ~SENSOR_PARED_ECHO;

	//Configuro los puertos encendido como salida.
	DDRA |= (1<<PA0);

}


uint32_t medicion_distancia_pared(uint8_t sensor){

    uint32_t    cuentas = 0, cuentas_inicial, cuentas_final;

    //Desactivo INT 2 al medir, para evitar problemas.
    GICR &= ~(1<<INT2);

    while(cuentas == 0){

        //pongo en 1 el trigger
        PORTA |= sensor;

        //Espero 10 us
        _delay_us(10);

        //pongo en 0 el trigger
        PORTA &= ~sensor;

        //Espero a que el echo sea 1
        while((PINB & SENSOR_PARED_ECHO) != SENSOR_PARED_ECHO);

        interrupciones_timer_1 = 0;
        cuentas_inicial = TCNT1;
        status_flag = 0;
        sensor_active = 1;

        //Espero a que el echo sea 0
        while( ((PINB & SENSOR_PARED_ECHO) == SENSOR_PARED_ECHO) && (status_flag == 0) );
        sensor_active = 0;

        cuentas_final = TCNT1;

        cuentas = (256 - cuentas_inicial) + 256 * interrupciones_timer_1 + cuentas_final;

    }

    //Vuelvo a activar la interrupcion INT2
    GICR |= (1<<INT2);


    if (status_flag == 1)
        return 0xFFFFFFFF;


    else
        return cuentas;
}
\end{lstlisting}

\subsection{sensoresPared.h}
\begin{lstlisting}

#ifndef _SENSORES_PARED_H_
#define _SENSORES_PARED_H_

#include <avr/io.h>
#include <util/delay.h>

#define	SENSOR_PARED_IZQ	(1 << PA3)
#define	SENSOR_PARED_DER	(1 << PA4)
#define	SENSOR_PARED_CEN	(1 << PA5)
#define SENSOR_PARED_ECHO	(1 << PB2)

#define CANTIDAD    10

#endif

//Definicion de prototipos
inline void apagar_timer(void);
inline void	encender_timer(void);
void        inicializar_timer(void);
void        inicializar_puertos_sensores_pared(void);
uint32_t    medicion_distancia_pared(uint8_t);

\end{lstlisting}

\subsection{sensoresPiso.c}
\begin{lstlisting}

#include "sensores_piso.h"


/*los tres sensores del piso estan conectados a la misma interrupcion externa INT2
y a su vez a PA0=izq PA1=der y PA2=centro
La idea con este codigo es primero inicializar las interrupciones por flanco ascendente (usando algun tipo de debouncer  para evitar efecto rebote)
y poder identificar que interrupcion fue activada para luego poder usarla en el codigo del robot.
*/

inline void encender_sensores_piso(void){

    PORTC |= (1<<1);

}


inline void apagar_sensores_piso(void){

    PORTC &= ~(1<<1);

}


void inicializar_sensores_piso(void)
{

    //Configuro el pin de los CNY70 como salida.
    DDRC    |= (1<<1);

    //configuro el pin de interrupcion como entrada (no se si es necesario)
    DDRB    &= ~(1<<2);
    PORTB   &= ~(1<<2);

    // activamos la interrupcion INT2 por flanco ascendente ISC2=1
    MCUCSR |= (1<<ISC2);

    //configuramos General Interrupt Control Register
    //activamos la interrupcion INT2=1
    GICR |= (1<<INT2);

}
\end{lstlisting}

\subsection{sensoresPiso.h}
\begin{lstlisting}
#ifndef _SENSORES_PISO_H
#define _SENSORES_PISO_H


#include <avr/io.h>
#include <avr/interrupt.h>

#endif


//prototipos de funcion.
inline void encender_sensores_piso(void);
inline void apagar_sensores_piso(void);
void        inicializar_sensores_piso(void);
\end{lstlisting}

\end{document}
